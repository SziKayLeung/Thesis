\begin{abstract}

Alzheimer's disease is a devastating neurodegenerative disorder characterised by progressive intracellular accumulation of hyperphosphorylated tau and extracellular deposition of amyloid-beta. It affects over 50 million people worldwide with numbers expecting to triple by 2050. Despite recent success in identifying genetic risk factors for AD, the mechanisms underpinning disease progression remain unknown. There is increasing evidence for altered transcriptional regulation and RNA splicing in the development of AD pathology. However, current studies exploring isoform diversity in the AD brain are constrained by the inherent limitations of standard short-read RNA-sequencing approaches, which fail to capture full-length transcripts critical for transcriptome assembly.  

The primary aim of this thesis was to utilise two long-read sequencing approaches, Pacific Biosciences isoform sequencing and Oxford Nanopore Technologies nanopore cDNA sequencing, to examine isoform diversity and transcript usage in the cortex, and identify alternative splicing events associated with AD pathology in a transgenic model of tau pathology (rTg4510). By generating long reads that span full-length transcripts, our studies revealed widespread RNA isoform diversity with unprecedented detection of novel transcripts not present in existing genome annotations. We further performed ultra-deep targeted long-read sequencing of 20 AD-risk genes, identifying robust expression changes at the transcript level associated with tau accumulation in the cortex. Our analyses provide a systematic evaluation of transcript usage, even in the absence of gene-level expression alterations, and highlight the importance of alternative RNA splicing as a mechanism underpinning gene regulation in the development of tau pathology. 

Finally, this thesis presents a laboratory and bioinformatics pipeline for the systematic analysis of isoform diversity and alternative splicing using long-read sequencing. The data generated as part of this research have implications for our understanding of the mechanisms driving the development of tau pathology, and represent a valuable resource to the wider research community.

\end{abstract}