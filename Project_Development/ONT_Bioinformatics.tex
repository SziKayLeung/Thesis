\documentclass[../Main/Knit.tex]{subfiles}
\newpage
\subsection{Bioinformatics Pipeline}

\begingroup
\parindent=0em
\etocsettocstyle{\rule{\linewidth}{\tocrulewidth}\vskip0.5\baselineskip}{\rule{\linewidth}{\tocrulewidth}}
\setcounter{tocdepth}{3}
\etocsetnexttocdepth{5}
\localtableofcontents 
\endgroup

\subsubsection{Base-calling}
The first analysis is to convert or "base-call" the electrical signals to the corresponding bases using Albacore, or a more recently developed package, Guppy, that requires information on the: 
\begin{enumerate}
	\item Chemistry of the run such as whether 1D or 1D\textsuperscript{2}
	\item Flow cell version used, to define the protein nanopore and subsequent 6-mer, which has different residual current	
	\item Sequencing kit used as this specifies the translocation speed, which informs the event segmentation algorithm how to recognise the corresponding bases from the electrical signal
	\item use of barcoding to run multiple samples in one flow cells for downstream demultiplexing
	\item type of output file, such as FASTQ or fast5
\end{enumerate}
In contrast to PacBio's SMRT with the ability to generate consensus long reads, the raw accuracy of nanopore 1D cDNA sequencing is relatively low between 85–87\%; however, significant improvements are made on reducing error rate by rapid development of both the technology and library preparation methods (Volden et al. 2018). Such high error rates, from frequent base deletions and insertions particularly near splice sites, can result in spurious alignments and in correct clustering of reads. 


\subsubsection{Quality Control of Run and Base-called Reads}
There are a number of developed bioinformatic tools that provide a quality assessment of base-called reads, which provide information on: 
\begin{enumerate}
	\item Performance of the sequencing run for each flow cell
	\item Distribution of base called read lengths
	\item Distribution of quality scores: over base pair per read, over time across all reads across the flow cell 
	\item No of reads generated over time
\end{enumerate}

\subsubsection{Filtering of Base-called Reads}
Base-called reads are filtered using NanoFilt (part of Nanopack) based on the following parameters: 
\begin{enumerate}
	\item Filter on a minimum average read quality score of 7  
	\item No Filter on a minimum read length  
	\item No Filter on a maximum read length
\end{enumerate}

\subsubsection{Removing of Nanopore and cDNA sequencing adapters}
Similarly to the Iso-Seq protocol, nanpore ligation adapters and cDNA sequencing primer sequences are removed to prevent spurious alignment, using PoreChop. This tool finds and trims adapters at the end of reads, and remove any "chimeric" reads with adapters in the middle. As detailed in WTAC's course, the nanopore.read.py script was edited to generate the output file for downstream analyses, and the adapters.py was amended to include the specific adapter sequence in accordance to developer's instructions. 

\
\begin{lstlisting}[language=bash, caption= PoreChop command]
python porechop-runner.py 
--format fastq \ # input format
-pass_reads.fq \ # input reads
-o pass_reads_choped.fq \ # output trimmed reads

# Additional parameters to run porechop
# Discard reads with adapters in the middle, i.e. "chimeric reads"
# An adapter would only be trimmed if it has 90% identity (default)
# Report for presence and location of adaptors for every read as output_adaptor_alignments_stats
--discard_middle \		 
--adapter_threshold 90
--verbosity 4 \ 	

# Parameters to control trimming of adaptors from read ends 
# 100 bases at each end of reads are searched for adaptor
# Adapter alignments smaller than 15bases will be ignored
# 1 base removed next to adapters found at the ends of reads
# Adapters at ends of reads must have >75% identity before removed

--end_size 100 \			
--min_trim_size 15 \		
--extra_end_trim 1 \	
--end_threshold 75 \		
> output_adaptor_alignment_stats
\end{lstlisting}

\subsubsection{Definition of sequence adapters for removal}
After cDNA synthesis with SMARTer PCR cDNA synthesis kit (ClonTech), and ligation of ONT adapters, cDNA have the structure depicted in Figure X. Due to the nature of cDNA sythesis primers and ONT adapter sequences, it is able to:
\begin{enumerate}
	\item identity the beginning (Transcription Start Site - TSS) and the end (Transcription End Site - TES) of the cDNA sequence
	\item differentiate between the plus strand of the cDNA (strand corresponding to the original mRNA sequence with a poly-A tail ) and the negative strand (complementary strand to original mRNA sequence with a poly-T tail)
\end{enumerate}

The motor protein binds to either 5' end of the cDNA strand for translocation through the pore. The sequence of the ONT adapters used in the SQK-LSK109 kit are recorded in Table X. 

\subsubsection{Filtering and processing of trimmed reads}
The trimmed reads from Porechop contains both plus and negative strands, and only reads containing at least one adapter at either end are retained to maximise read usage. However, it is important to be more stringent for isoform identification and quantification, and retain reads with both adapters at the end. It is further necessary to reverse complement the reads that correspond to the negative strands; all the reads would then start with the TSS and end with the polyA sequence. Using cutadapt, the polyA sequence are then trimmed with cutadapt with a window gap of 40 bases from the end (as polyT primer in cDNA synthesis introduces 30As and to avoid spurious base calling).

\subsubsection{Genome Alignment}
Similarly to Iso-Seq, trimmed reads are then aligned to the reference genome (as opposed to transcriptome as otherwise miss novel isoforms) using splice-aware aligner Minimap. However, unlike the Iso-Seq pipeline which generates high-quality transcripts, ONT reads are more prone to errors and having a lower mapping coverage (check). The reads from Minimap2 are thus further filtered based on alignment quality (identity and percentage of alignment length) using htsbox.

\subsubsection{Transcript Collapse}
Similarly to Iso-Seq, aligned and filtered reads need to be collapsed to generate unique transcripts. This is done with Cupcake in Iso-Seq, and TAMA collapse for ONT with the below parameters \ref{Tama_collapse_command}. The unique transcripts are then merged with the reference genome using TAMA merge for annotation and further characterisation. TAMA collapse and merge can also be performed for Iso-Seq reads, however, this is not part of the official pipeline. 

\
\begin{lstlisting}[language=bash, caption= Tama collapse and merge command, label= Tama_collapse_command]
python tama_collapse.py \
-s aligned.sam \ 	# input file
-f mm10.fa \		# genome fasta sequence
-p tama_collapse_output \

# Parameters to control collapse of transcripts by: 
# common exon start/end site rather than the longest feature
# using only reads where 95% are mapped to the genome
# 80% identity
# Difference of 50 bases at 5' and 3' end for reads to be collapsed 
# Difference of 20 bases at splice junctions for reads to be collapsed
# Merge duplicate transcript groups 

-e common_ends -c 95 -i 80 -x capped -a 50 -z 50 -m 20 -d  merge_dup 

python tama_merge.py -f file.list -a 50 -z 50 -m 20 -p tama_merge_output
\end{lstlisting}

\uline{\textbf{FLAIR}}: Full-Length Alternative Isoform analysis of RNA  (FLAIR\nomenclature{FLAIR}{Full-Length Alternative Isoform analysis of RNA}) 
Three steps are involved: Correct splice sites with short reads if incorrect splice site is within 10base pairs away from correct splice site, collapse reads to generate consensus sequences. This involves first grouping reads with identical splice junctions - "first pass nanopore isoform transcriptome"; the representative isoform within each group is determined by the most supported transcription and end site. All the reads, including reads that were aligned but not able to be fully corrected, are re-aligned to the "first-pass isoform" with the best alignment. First-pass isoforms that have fewer than three supporting reads are filtered out; three supporting reads selected as threshold as this gave the highest base sensitivity without compromising on precision.  

\subsubsection{Isoform Quantification} 
In contrast to Iso-Seq, isoform quantification from ONT is relatively simpler in that each nanopore read corresponds to a single transcript (Tang et al. 2020). However, ambiguity still remains with assignment of truncated reads 

\subsubsection{Limitations of Oxford Nanopore} 
Refer to \cite{Workman2019a} for information on strand break etc