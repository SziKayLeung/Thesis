\documentclass[../Main/Knit.tex]{subfiles}
\newpage

\subsection{Lab Pipeline}
\begingroup
\parindent=0em
\etocsettocstyle{\rule{\linewidth}{\tocrulewidth}\vskip0.5\baselineskip}{\rule{\linewidth}{\tocrulewidth}}
\etocsetnexttocdepth{5}
\localtableofcontents 
\endgroup

\subsubsection{cDNA synthesis}
For a fair, direct comparison between ONT's MinION sequencing and PacBio's IsoSeq, 200ng total RNA extracted using AllPrep DNA/RNA Mini Kit (Qiagen) was likewise converted to single-stranded DNA using SMARTer PCR cDNA Synthesis(ClonTech).

\subsubsection{ONT MinION Library Preparation}
Despite a range of protocols available on the Oxford Nanopore community protocol that can be used pending on the source of sample, the SQK-LSK109 kit was used with cDNA as starting material. This kit is PCR-free and as such is dependent upon generation of high-quality and full-length cDNA, which would be provided using the SMARTer PCR cDNA synthesis kit rather than that detailed in 1D Strand switching cDNA by ligation protocol (SQK-LSK108).

\subsubsection{Repair DNA and Ends}
DNA calibration strand (DCS) is 3.6kb amplicon of Lambda genome, and is included in the sample library as a quality control of base-calling and sample preparation. End Repair prepare the ends of cDNA molecules for adapter attachment by addition of dA nucleotides

\subsubsection{Adapter Ligation}
Post DNA repair and end repair, ONT adapters with dT overhang are ligated to the 5’ end of the dA-tailed cDNA molecules by hybridisation. The ONT adapters contain: 
\begin{itemize}
	\item motor protein (loaded processive enzyme?) that can bind to the nanopore and control/increase the speed of DNA translocation through the pore. While it is active in solution, it is inhibited from contacting the rest of the DNA through specialised bases in the adapter.
	\item cholesterol tether to facilitate DNA capture as (1) tethers the DNA molecule to the lipid bilayer (membrane) of the flow cell  (2) reduces amount of diffusion of the DNA molecule from three dimensions (i.e the volume of whole buffer) to two dimensions (i.e. across the lipid bilayer)  
\end{itemize}

\subsubsection{Priming the Flow Cell}
Sequencing buffer provides the optimal chemical conditions for powering DNA translocation through the Nanopore. This is the substrate cofactor of the motor enzyme that is used for DNA translocation process in the pore. 
